\documentclass{article}

\usepackage{arxiv}

\usepackage[utf8]{inputenc} % allow utf-8 input
\usepackage[T1]{fontenc}    % use 8-bit T1 fonts
\usepackage{hyperref}       % hyperlinks
\usepackage{url}            % simple URL typesetting
\usepackage{booktabs}       % professional-quality tables
\usepackage{amsfonts}       % blackboard math symbols
\usepackage{nicefrac}       % compact symbols for 1/2, etc.
\usepackage{microtype}      % microtypography
\usepackage{cleveref}       % smart cross-referencing
\usepackage{graphicx}
\usepackage{natbib}
\usepackage{doi}

\title{A meta-introspector template for the \emph{arxiv} style}

% Here you can change the date presented in the paper title
\date{May 27, 2024}
%\date{}

\newif\ifuniqueAffiliation
% Comment to use multiple affiliations variant of author block 
\uniqueAffiliationtrue

\ifuniqueAffiliation % Standard variant of author block
\author{ \href{https://github.com/meta-introspector/time/blob/main/authors/james-michael-dupont.org}{\includegraphics[scale=0.06]{metaintrospector-icon.png}\hspace{1mm}Mike DuPont}\thanks
	Author\\
	Meta Introspector\\
	\texttt{jmikedupont2@gmail.com} \\
	%% examples of more authors
	%% \AND
	%% Coauthor \\
	%% Affiliation \\
	%% Address \\
	%% \texttt{email} \\
	%% \And
	%% Coauthor \\
	%% Affiliation \\
	%% Address \\
	%% \texttt{email} \\
	%% \And
	%% Coauthor \\
	%% Affiliation \\
	%% Address \\
	%% \texttt{email} \\
}
\else
% Multiple affiliations variant of author block
\usepackage{authblk}
\renewcommand\Authfont{\bfseries}
\setlength{\affilsep}{0em}
% box is needed for correct spacing with authblk
\newbox{\orcid}\sbox{\orcid}{\includegraphics[scale=0.06]{metaintrospector-icon.png}} 
\author[1]{%
  \author{ \href{https://github.com/meta-introspector/time/blob/main/authors/james-michael-dupont.org}{\includegraphics[scale=0.06]{metaintrospector-icon.png}\hspace{1mm}Mike DuPont}\thanks\texttt{jmikedupont2@gmail.com}}}%
}
\affil[1]{Author, Meta Introspector}

\fi

% Uncomment to override  the `A preprint' in the header
\renewcommand{\headeright}{Technical Report}
\renewcommand{\undertitle}{Technical Report}
\renewcommand{\shorttitle}{Meta Introspector \textit{arXiv} Template}

%%% Add PDF metadata to help others organize their library
%%% Once the PDF is generated, you can check the metadata with
%%% $ pdfinfo template.pdf
\hypersetup{
pdftitle={A meta introspector template for the arxiv style},
pdfsubject={q-bio.NC, q-bio.QM},
pdfauthor={James Michael DuPont},
pdfkeywords={usable, simple, unified, audited, executable, translatable, extractable, comprehesive, self-modifying, self-executing, self-bootrstrapping, self-testing, self-proving, self-improving, self-documenting, reproducible template, introspector},
}

\begin{document}
\maketitle

\begin{abstract}
	A template for a self improving system.
\end{abstract}


% keywords can be removed
\keywords{usable \and simple \and unified \and audited \and executable \and translatable \and ex tractable \and comprehensive \and self-modifying \and self-executing \and self-bootstrapping \and self-testing \and self-proving \and self-improving \and self-documenting \and reproducible template \and introspect-or}

\section{Introduction}

The introspect-or is a long term research project from the author to inspect and understand the internals of the computation and how they relate to the meta narrative surrounding them.

\section{Concepts}
\label{sec:concepts}

\subsection{Concept of meme}
The idea of a meme, as a unit of mimicry, or copying of behavior.
Observed behavior is copied so that the result has the same effect.
The DNA of the meme can be seen as the vector embedding in a large language model
of the higher dimensional concepts behind the meme that unify the different expressions of it.

\subsection{Concepts of meme invocation}

The Running of a program can be seen as the invocation or the copying of a meme
and allowing it to execute.

\subsection{Concepts of context}
The context of execution is imagined to be in a closed world model,
inside of a emacs editor, on a Linux server, running Debian,
inside of a reproducible Guix bootstrap,
with all source code for all tools available,
with root access,
with enough storage to sample every instruction using Linux perf recording,
in a secure and audited environment, like a DAO that is operating in a transparent manner.

This context can be seen as a grand tapestry or tape, a woven fabric of threads
spun by the three sisters of fate that create the thread of life.

\subsection{Concepts of narrative}

The narrative of the execution can be seen as the binary instructions to be executed, or the story about what is to be done. We can imagine that
we have a narrative that connects what we are doing and observing
to the tapestry. 

This connection from content to the narrative is interesting,
is means we can look up from our work and see the tapestry on the wall and try and follow the instructions. This is like looking up the next steps.

\subsection{Concepts of bootstrap}

The system being applied to itself to bootstrap itself is key.
We can think of this as part of the closed world that creates the world,
all of those parts are fixed in a way, they can change but it will create a different system. We can think of multiple bootstraps leading to similar equivalent systems.

\subsection{Concepts of meta-narrative}

The meta-narrative is a narrative about the narrative. We can think of this as a higher order topological space that creates and connects the parts of the narrative together. This allows for reflection, quasi quotation, splicing and mutation of the narrative using another narrative. 

\subsection{Combination of narrative and context}

The context as we said is the surrounding of the current point in time,
the things leading up to now, the influences, the state of the  variables,
the state of the system, the functions that created that state. The narrative should be connected and proven to be connected to the state, thus explainable.

\subsection{Concept of operating system}
The operating system manages the execution context of the processes
allocating resources and protecting them.

\subsection{Concept of Context of execution}

The context of execution we mentioned contains the features we sample of the
environment at a given time and how they can be related to the narrative of the purpose.

\subsection{Concept of Time}

Time emerges as a series of samples, appended to the collected data-set
on a device in the context of execution of a program.

\subsection{Concepts of execution}
The concept of execution is the following of
the instructions.

\subsubsection{The concept of the instruction}
the instruction can be seen as a selection of features assigned to irreducible factors such as prime numbers in such a way that we can efficiently do a symbolic regression on them, we can consider the instruction to be the prime factorization of a part of the system if expressed as a Godel number.

\paragraph{Paragraph}

The tracing of the execution of instructions over time
creates a set of values of registers and modifies the states
of object in memory or on device as side effects.
We can model and prove these side effects.

\paragraph{Paragraph}

The augmentation of the instructions with code
to capture the results, construct proofs and annotate the execution
is one useful modification of the code. 

So in that way the instructions become templates to be interpreted.

We can see the block or set of instructions as having a desired effect
and we try and model it. 

\bibliographystyle{unsrtnat}
\bibliography{references} 
\end{document}
                                                                                                                                                                                                                                                                                                                                           
